\documentclass[conference]{IEEEtran}

\pagestyle{plain}

\usepackage{cite}
\usepackage{amsmath,amssymb,amsfonts}
\usepackage{algorithmic}
\usepackage{graphicx}
\usepackage{textcomp}
\usepackage{xcolor}
\usepackage{xspace}
\usepackage{listings}
\usepackage{lmodern}

% (2) specify encoding
\usepackage[T1]{fontenc}

% (3) load symbol definitions
\usepackage{hyperref}

%%%%%%%%%%%%%%%%%%%%%%%%%%%%%%%%%%%%%%%%%%%%%%%%%%%%%%%%%%%%%%%%%%%%%

% For autoref naming of sections
\renewcommand{\sectionautorefname}{Section}
\renewcommand{\subsectionautorefname}{Section}
\renewcommand{\subsubsectionautorefname}{Section}

% Note/todo commands
\newcommand{\needref}{\textcolor{blue}{[ref?]}}
\newcommand\grant[1]{{\color{purple}[Grant: #1]}}

% Listing configuration
\definecolor{comments}{RGB}{74,131,31}
\definecolor{strings}{RGB}{0,128,64}
\definecolor{numbers}{RGB}{44,45,211}
\definecolor{identifiers}{RGB}{153,153,0}

\lstset{
  language=C,
  numbers=none,
  basicstyle={\tt\small},
  stringstyle=\color{strings},
  commentstyle=\color{comments},
  keywordstyle={\color{blue}\bfseries},
  %identifierstyle={\ttfamily\color{identifiers}},
  emph      = [1]{
    challengeFunction,secretFunction},
  emphstyle=[1]{\ttfamily\bfseries\color{blue}},
  stepnumber=1,                   % the step between two line-numbers.
  numbersep=10pt,                  % how far the line-numbers are from the code
  backgroundcolor=\color{white},  % choose the background color. You must add \usepackage{color}
  showspaces=false,               % show spaces adding particular underscores
  showstringspaces=false,         % underline spaces within strings
  showtabs=false,                 % show tabs within strings adding particular underscores
  tabsize=2,                      % sets default tabsize to 2 spaces
  captionpos=b,                   % sets the caption-position to bottom
  breaklines=true,                % sets automatic line breaking
  breakatwhitespace=true,         % sets if automatic breaks should only happen at whitespace
  xleftmargin=2em,framexleftmargin=1.5em
}%

\lstdefinelanguage{none}{
  identifierstyle=
}

%%%%%%%%%%%%%%%%%%%%%%%%%%%%%%%%%%%%%%%%%%%%%%%%%%%%%%%%%%%%%%%%%%%%%

\begin{document}

\title{Kernel Sanders: CSAW {ESC'19} Quals\\
}

\author{\IEEEauthorblockN{
Grant Hernandez\IEEEauthorrefmark{1},
Claire Seiler\IEEEauthorrefmark{1},
Owen Flannagan\IEEEauthorrefmark{1},
Hunter Searle\IEEEauthorrefmark{1},
Kevin R.B. Butler\IEEEauthorrefmark{1}}
\IEEEauthorblockA{\IEEEauthorrefmark{1}University of Florida, Gainesville, FL, USA\\ \{grant.hernandez,
  cseiler, owenflannagan, huntersearle, butler\}@ufl.edu}
}

\maketitle

%\begin{abstract}
%\end{abstract}

\section{Introduction}

The advent of Radio Frequency Identification (RFID) technology has quickly been followed by its rapid and pervasive integration across multiple industries, from transportation to security. However, with its widespread adoption comes an increased need to investigate the potential security implications of integrating RFID technology. As RFID is commonly utilized for a wide variety of authentication, access control, asset tracking, payment, and identification applications, these systems could be vulnerable to attacks that exploit the underlying RFID technology. 

RFID systems typically consist of three main components: an RFID tag, an RFID reader, and an antenna. A reader, which is a two-way radio transmitter-receiver, sends radio frequency signals to tags and reads the response. Tags, which store data like serial numbers, can be read-only or read/write. Additionally, the tags may also be designated as passive or active, meaning they are powered by the radio energy transmitted by the reader or by an on-board battery, respectively.

These systems raise a variety of security concerns; they are potentially vulnerable to a variety of eavesdropping, spoofing, or jamming methods. Additionally, common reverse engineering and firmware exploitation techniques can be applied to the exploitation of RFID readers. 

RFID readers, like other embedded devices, contain firmware in non-volatile, flash memory. This firmware can be extracted from a physical memory chip using tools like \textit{flashrom}, \textit{binwalk}, Bus Pirates, and logic analyzers. Alternatively, if available, JTAG/SWD could be used to perform a memory dump of the running CPU if accessing the firmware via NOR/NAND flash is too difficult. After extraction, static and dynamic analysis can be done to identify potential vulnerabilities that could lead to compromise of the reader. In static analysis, disassemblers like GHIDRA, IDA Pro, radare2, or Binary Ninja can be leveraged to analyze the assembly instructions corresponding to the firmware. A decompiler can assist in understanding and to recreate the source code in a high level language.

Once disassembled or decompiled, specific vulnerabilities can be identified and targeted exploits can be developed. Vulnerabilities like stack- and heap-based buffer overflows, off-by-one errors, integer overflows, uncontrolled format strings, poor input validation and sanitization, OS command injection, disabled (but not removed) debugging functionality, and hardcoded credentials often plague embedded firmware, which typically rely on languages with manual memory management like C or C++. Thus, these types of vulnerabilities can serve as a guide to analyzing the disassembled firmware of an RFID reader. 

After identifying a vulnerability, a targeted exploit can be created. Mitigations like stack canaries, heap protection, ARM's specific eXecute Never (XN), RELocation Read-Only (RELRO), Position-Independent Executable (PIE), and Address Space Layout Randomization (ASLR) can be overcome with some ingenuity and techniques like return-oriented programming (ROP) chaining, stack smashing, heap spraying, information disclosures and more. Exploit writing frameworks, such as pwntools, can assist in developing an exploit for an RFID reader's firmware, depending on the device architecture and the protections enabled.


\section{Challenge}
To begin our analysis of the given \texttt{qualification.out} object, we start by running the GNU \texttt{file} command on it.
\begin{lstlisting}[numbers=left,language=none]
qualification.out: ELF 64-bit LSB executable, x86-64, ... , not stripped
\end{lstlisting}
Immediately we know that this is an x86-64 ELF binary executable, which is unstripped, meaning functions should have names.
Next running \texttt{strings} on the binary (``...'' means snipped text) we see:
\begin{lstlisting}[numbers=left,language=none]
...
Great Job! The flag is what you entered
The flag is <<shhimhiding>>
;*3$"
GCC: (Ubuntu 4.8.4-2ubuntu1~14.04.4) 4.8.4
...
qualification.cpp
...
_Z14secretFunctionv
...
_Z17challengeFunctionPc
\end{lstlisting}
From the strings, we see a ``good flag'' message, an actual flag, that this binary was written as C++, and two C++ mangled functions.

With initial static analysis out of the way, we can set the file as executable and do some dynamic analysis.

\begin{lstlisting}[language=none]
$ chmod +x qualification.out
$ ./qualification.out
$ ./qualification.out test
$ ./qualification.out shhimhiding
\end{lstlisting}

Running the binary with and without arguments (even the flag found via strings) yields no ``goodboy'' message. To investigate further, we start GHIDRA 9.0 to begin our analysis.
We create a new GHIDRA project and load the binary into it. We open the CodeBrowser tool and perform auto-analysis.

There are three challenges therefore we shall go through this binary one challenge at a time. The first step we took was to look at the main function. The main frunction in this binary siomply printed out the menu asking for the number of the challenge the user wanted to try out. Once a number is selected the main function calls \texttt{challenge\char`_x} where "x" is the number input. Ghidra outputs the following for \texttt{challenge\char`_1}.

\begin{lstlisting}
undefined4 challenge_1(char *param_1)

{
	undefined4 uVar1;
	char *local_24;
	undefined4 local_20;
	undefined4 local_1c;
	undefined4 local_18;
	int local_14;
	
	local_20 = 0x2d435345;
	local_1c = 0x30323032;
	local_18 = 4;
	local_24 = param_1;
	if (((*param_1 == 'C') && (param_1[1] == 'S')) && (param_1[2] == 'E')) {
	local_14 = 3;
	while (local_14 < 0xb) {
		if (local_14 + -3 + (uint)(byte)param_1[local_14] !=
			(uint)*(byte *)((int)&local_24 + local_14 + 1)) {
		puts("Incorrect!");
		return 0xfffffffe;
		}
		local_14 = local_14 + 1;
	}
	puts("Correct, you solved Challenge 1!");
	uVar1 = 0;
	}
	else {
	puts("Incorrect!");
	uVar1 = 0xffffffff;
	}
	return uVar1;
}
\end{lstlisting}

The first thing we noticed about this challenge is that it pushed several bytes onto the stack. Decoding these as ascii text revealed a secret phrase "ESC-2020". The function then has a simple check to ensure that the first three characters of the input is "CSE". After that, it loops over the remaining 8 character of the input. Each character has the value (i-3) added to it, where i is the loop index, then it is compared to a value stored on the stack, with an offset of (i+3). Checking the stack, we can see that the values it is compared against are the secret phrase. Thus it is a simple task of subtracting (i-3) from each character in the phrase to arrive at the correct input, using the following python code:

\begin{lstlisting}
password = ESC-2020
i = 3
for char in password:
	print(chr(ord(char) + (i-3)))
	i += 1
\end{lstlisting}

Combining the output of the above code with "CSE" gives the input \texttt{CSEERA*.+,)}.

Challenge 2 prints the following.

\begin{lstlisting}
undefined4 challenge_2(int param_1)

{
	undefined4 local_24;
	undefined4 local_20;
	undefined4 local_1c;
	undefined4 local_18;
	uint local_14;
	
	local_24 = 0x7a707a65;
	local_20 = 0x6f6d656c;
	local_1c = 0x7571736e;
	local_18 = 0x797a6565;
	local_14 = 0;
	while( true ) {
	if (0xf < local_14) {
		puts("Correct, you solved Challenge 2!");
		return 0;
	}
	if ((*(byte *)((int)&local_24 + local_14) ^ 0x2a) != *(byte *)(local_14 + param_1)) break;
	local_14 = local_14 + 1;
	}
	puts("Incorrect!");
	return 0xffffffff;
}
\end{lstlisting}

It is clear that there is a simple xor of each char in 'ezpzlemonsqueezy' with 0x2a. This produces \texttt{OPZPFOGEDY[\char`_OOPS} which is the valid resulting key.

Challenge 3 seemed similar to the previous 2, with a for loop iterating over each character of the input, altering it, then comparing it to a set value. However, we found that the comparison phrase was first sent to another function arx(). This function took as inputs the secret phrase and the length of the input. It then went through several rounds of transformations detailed below.

\begin{enumerate}
\item initial code is ezpzlemonsqueezy
\item Do the following ten times
	\begin{itemize}
	\item add 0x11 to each byte
	\item shift each char to the left(with wraparound), and with 0xff
	\item xor each char with 0xd
	\end{itemize}
\item Take unsigned mod of each char by 0x3c, and with 0xff
\item Add 0x22 to each char
\end{enumerate}

This process can be visualized in the challenge code below.

\begin{lstlisting}
undefined4 challenge_3(int param_1)

{
	undefined4 local_24;
	undefined4 local_20;
	undefined4 local_1c;
	undefined4 local_18;
	uint local_14;
	
	local_24 = 0x7a707a65;
	local_20 = 0x6f6d656c;
	local_1c = 0x7571736e;
	local_18 = 0x797a6565;
	arx(&local_24,0x10);
	local_14 = 0;
	while( true ) {
	if (0xf < local_14) {
		puts("Correct, you solved Challenge 3!");
		return 0;
	}
	if (*(char *)((int)&local_18 + (3 - local_14)) != *(char *)(local_14 + param_1)) break;
	local_14 = local_14 + 1;
	}
	puts("Incorrect!");
	return 1;
}
\end{lstlisting}

After these transformations, the input is compared, in reverse error to the phrase returned by arx(). To solve the challenge, we simply wrote a program that ran all of the steps, then reversed the output to get the correct input: I0E\char`;3.<2<3G<33C\char`#

\begin{lstlisting}
#include "stdio.h"

int main(){
int code[0x10] = {0x65,0x7a,0x70,0x7a,0x6c,0x65,0x6d,0x6f,0x6e,0x73,0x71,0x75,0x65,0x65,0x7a,0x79};
for(int i = 10; i > 0; i--){
	for (int a = 0; a < 0x10; a++){
		code[a] = (code[a] + '\x11') & 0xff;
	}
	int first = code[0];
	for(int b = 0; b < 0xf; b++){
		code[b] = code[b+1];
	}
	code[0xf] = first;
	for(int c = 0; c < 0x10; c++){
		code[c] = code[c] ^ 0xd;
	}

}
for(int d = 0; d < 0x10; d++){
	code[d] = ((unsigned)code[d] % 0x3c) & 0xff;
	code[d] = code[d] + 0x22;
}
for(int out = 0xf; out >= 0; out--){
	printf("%c", code[out]);
}
printf("\n");
}
\end{lstlisting}

\section{Conclusion}
To complete this qualifier challenge, our team used GHIDRA, a reverse engineering tool, along with a riscv ghidra processor module to work backwards through an unfamiliar binary file to obtain the correct flag. For the first challenge, we collected the flag \texttt{CSEERA*.+,)}. For the second, the flag \texttt{OPZPFOGEDY[\char`_OOPS} was quickly obtained. For the final challenge, we built off of knowledge gained from the previous challenge and produced the flag I0E\char`;3.<2<3G<33C\char`#. From the experiences gained through this qualifier, we feel well-adjusted to proceed with the firmware-level analysis and exploitation of different RISC-V programs.

%\section*{References}
%
%EXAMPLE: Please number citations consecutively within brackets \cite{b1}.
%
%\begin{thebibliography}{00}
%\bibitem{b1} G. Eason, B. Noble, and I. N. Sneddon, ``On certain integrals of Lipschitz-Hankel type involving products of Bessel functions,'' Phil. Trans. Roy. Soc. London, vol. A247, pp. 529--551, April 1955.
%\end{thebibliography}
%\vspace{12pt}

\end{document}
